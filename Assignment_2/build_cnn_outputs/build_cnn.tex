\documentclass[11pt]{article}

    \usepackage[breakable]{tcolorbox}
    \usepackage{parskip} % Stop auto-indenting (to mimic markdown behaviour)
    

    % Basic figure setup, for now with no caption control since it's done
    % automatically by Pandoc (which extracts ![](path) syntax from Markdown).
    \usepackage{graphicx}
    % Keep aspect ratio if custom image width or height is specified
    \setkeys{Gin}{keepaspectratio}
    % Maintain compatibility with old templates. Remove in nbconvert 6.0
    \let\Oldincludegraphics\includegraphics
    % Ensure that by default, figures have no caption (until we provide a
    % proper Figure object with a Caption API and a way to capture that
    % in the conversion process - todo).
    \usepackage{caption}
    \DeclareCaptionFormat{nocaption}{}
    \captionsetup{format=nocaption,aboveskip=0pt,belowskip=0pt}

    \usepackage{float}
    \floatplacement{figure}{H} % forces figures to be placed at the correct location
    \usepackage{xcolor} % Allow colors to be defined
    \usepackage{enumerate} % Needed for markdown enumerations to work
    \usepackage{geometry} % Used to adjust the document margins
    \usepackage{amsmath} % Equations
    \usepackage{amssymb} % Equations
    \usepackage{textcomp} % defines textquotesingle
    % Hack from http://tex.stackexchange.com/a/47451/13684:
    \AtBeginDocument{%
        \def\PYZsq{\textquotesingle}% Upright quotes in Pygmentized code
    }
    \usepackage{upquote} % Upright quotes for verbatim code
    \usepackage{eurosym} % defines \euro

    \usepackage{iftex}
    \ifPDFTeX
        \usepackage[T1]{fontenc}
        \IfFileExists{alphabeta.sty}{
              \usepackage{alphabeta}
          }{
              \usepackage[mathletters]{ucs}
              \usepackage[utf8x]{inputenc}
          }
    \else
        \usepackage{fontspec}
        \usepackage{unicode-math}
    \fi

    \usepackage{fancyvrb} % verbatim replacement that allows latex
    \usepackage{grffile} % extends the file name processing of package graphics
                         % to support a larger range
    \makeatletter % fix for old versions of grffile with XeLaTeX
    \@ifpackagelater{grffile}{2019/11/01}
    {
      % Do nothing on new versions
    }
    {
      \def\Gread@@xetex#1{%
        \IfFileExists{"\Gin@base".bb}%
        {\Gread@eps{\Gin@base.bb}}%
        {\Gread@@xetex@aux#1}%
      }
    }
    \makeatother
    \usepackage[Export]{adjustbox} % Used to constrain images to a maximum size
    \adjustboxset{max size={0.9\linewidth}{0.9\paperheight}}

    % The hyperref package gives us a pdf with properly built
    % internal navigation ('pdf bookmarks' for the table of contents,
    % internal cross-reference links, web links for URLs, etc.)
    \usepackage{hyperref}
    % The default LaTeX title has an obnoxious amount of whitespace. By default,
    % titling removes some of it. It also provides customization options.
    \usepackage{titling}
    \usepackage{longtable} % longtable support required by pandoc >1.10
    \usepackage{booktabs}  % table support for pandoc > 1.12.2
    \usepackage{array}     % table support for pandoc >= 2.11.3
    \usepackage{calc}      % table minipage width calculation for pandoc >= 2.11.1
    \usepackage[inline]{enumitem} % IRkernel/repr support (it uses the enumerate* environment)
    \usepackage[normalem]{ulem} % ulem is needed to support strikethroughs (\sout)
                                % normalem makes italics be italics, not underlines
    \usepackage{soul}      % strikethrough (\st) support for pandoc >= 3.0.0
    \usepackage{mathrsfs}
    

    
    % Colors for the hyperref package
    \definecolor{urlcolor}{rgb}{0,.145,.698}
    \definecolor{linkcolor}{rgb}{.71,0.21,0.01}
    \definecolor{citecolor}{rgb}{.12,.54,.11}

    % ANSI colors
    \definecolor{ansi-black}{HTML}{3E424D}
    \definecolor{ansi-black-intense}{HTML}{282C36}
    \definecolor{ansi-red}{HTML}{E75C58}
    \definecolor{ansi-red-intense}{HTML}{B22B31}
    \definecolor{ansi-green}{HTML}{00A250}
    \definecolor{ansi-green-intense}{HTML}{007427}
    \definecolor{ansi-yellow}{HTML}{DDB62B}
    \definecolor{ansi-yellow-intense}{HTML}{B27D12}
    \definecolor{ansi-blue}{HTML}{208FFB}
    \definecolor{ansi-blue-intense}{HTML}{0065CA}
    \definecolor{ansi-magenta}{HTML}{D160C4}
    \definecolor{ansi-magenta-intense}{HTML}{A03196}
    \definecolor{ansi-cyan}{HTML}{60C6C8}
    \definecolor{ansi-cyan-intense}{HTML}{258F8F}
    \definecolor{ansi-white}{HTML}{C5C1B4}
    \definecolor{ansi-white-intense}{HTML}{A1A6B2}
    \definecolor{ansi-default-inverse-fg}{HTML}{FFFFFF}
    \definecolor{ansi-default-inverse-bg}{HTML}{000000}

    % common color for the border for error outputs.
    \definecolor{outerrorbackground}{HTML}{FFDFDF}

    % commands and environments needed by pandoc snippets
    % extracted from the output of `pandoc -s`
    \providecommand{\tightlist}{%
      \setlength{\itemsep}{0pt}\setlength{\parskip}{0pt}}
    \DefineVerbatimEnvironment{Highlighting}{Verbatim}{commandchars=\\\{\}}
    % Add ',fontsize=\small' for more characters per line
    \newenvironment{Shaded}{}{}
    \newcommand{\KeywordTok}[1]{\textcolor[rgb]{0.00,0.44,0.13}{\textbf{{#1}}}}
    \newcommand{\DataTypeTok}[1]{\textcolor[rgb]{0.56,0.13,0.00}{{#1}}}
    \newcommand{\DecValTok}[1]{\textcolor[rgb]{0.25,0.63,0.44}{{#1}}}
    \newcommand{\BaseNTok}[1]{\textcolor[rgb]{0.25,0.63,0.44}{{#1}}}
    \newcommand{\FloatTok}[1]{\textcolor[rgb]{0.25,0.63,0.44}{{#1}}}
    \newcommand{\CharTok}[1]{\textcolor[rgb]{0.25,0.44,0.63}{{#1}}}
    \newcommand{\StringTok}[1]{\textcolor[rgb]{0.25,0.44,0.63}{{#1}}}
    \newcommand{\CommentTok}[1]{\textcolor[rgb]{0.38,0.63,0.69}{\textit{{#1}}}}
    \newcommand{\OtherTok}[1]{\textcolor[rgb]{0.00,0.44,0.13}{{#1}}}
    \newcommand{\AlertTok}[1]{\textcolor[rgb]{1.00,0.00,0.00}{\textbf{{#1}}}}
    \newcommand{\FunctionTok}[1]{\textcolor[rgb]{0.02,0.16,0.49}{{#1}}}
    \newcommand{\RegionMarkerTok}[1]{{#1}}
    \newcommand{\ErrorTok}[1]{\textcolor[rgb]{1.00,0.00,0.00}{\textbf{{#1}}}}
    \newcommand{\NormalTok}[1]{{#1}}

    % Additional commands for more recent versions of Pandoc
    \newcommand{\ConstantTok}[1]{\textcolor[rgb]{0.53,0.00,0.00}{{#1}}}
    \newcommand{\SpecialCharTok}[1]{\textcolor[rgb]{0.25,0.44,0.63}{{#1}}}
    \newcommand{\VerbatimStringTok}[1]{\textcolor[rgb]{0.25,0.44,0.63}{{#1}}}
    \newcommand{\SpecialStringTok}[1]{\textcolor[rgb]{0.73,0.40,0.53}{{#1}}}
    \newcommand{\ImportTok}[1]{{#1}}
    \newcommand{\DocumentationTok}[1]{\textcolor[rgb]{0.73,0.13,0.13}{\textit{{#1}}}}
    \newcommand{\AnnotationTok}[1]{\textcolor[rgb]{0.38,0.63,0.69}{\textbf{\textit{{#1}}}}}
    \newcommand{\CommentVarTok}[1]{\textcolor[rgb]{0.38,0.63,0.69}{\textbf{\textit{{#1}}}}}
    \newcommand{\VariableTok}[1]{\textcolor[rgb]{0.10,0.09,0.49}{{#1}}}
    \newcommand{\ControlFlowTok}[1]{\textcolor[rgb]{0.00,0.44,0.13}{\textbf{{#1}}}}
    \newcommand{\OperatorTok}[1]{\textcolor[rgb]{0.40,0.40,0.40}{{#1}}}
    \newcommand{\BuiltInTok}[1]{{#1}}
    \newcommand{\ExtensionTok}[1]{{#1}}
    \newcommand{\PreprocessorTok}[1]{\textcolor[rgb]{0.74,0.48,0.00}{{#1}}}
    \newcommand{\AttributeTok}[1]{\textcolor[rgb]{0.49,0.56,0.16}{{#1}}}
    \newcommand{\InformationTok}[1]{\textcolor[rgb]{0.38,0.63,0.69}{\textbf{\textit{{#1}}}}}
    \newcommand{\WarningTok}[1]{\textcolor[rgb]{0.38,0.63,0.69}{\textbf{\textit{{#1}}}}}
    \makeatletter
    \newsavebox\pandoc@box
    \newcommand*\pandocbounded[1]{%
      \sbox\pandoc@box{#1}%
      % scaling factors for width and height
      \Gscale@div\@tempa\textheight{\dimexpr\ht\pandoc@box+\dp\pandoc@box\relax}%
      \Gscale@div\@tempb\linewidth{\wd\pandoc@box}%
      % select the smaller of both
      \ifdim\@tempb\p@<\@tempa\p@
        \let\@tempa\@tempb
      \fi
      % scaling accordingly (\@tempa < 1)
      \ifdim\@tempa\p@<\p@
        \scalebox{\@tempa}{\usebox\pandoc@box}%
      % scaling not needed, use as it is
      \else
        \usebox{\pandoc@box}%
      \fi
    }
    \makeatother

    % Define a nice break command that doesn't care if a line doesn't already
    % exist.
    \def\br{\hspace*{\fill} \\* }
    % Math Jax compatibility definitions
    \def\gt{>}
    \def\lt{<}
    \let\Oldtex\TeX
    \let\Oldlatex\LaTeX
    \renewcommand{\TeX}{\textrm{\Oldtex}}
    \renewcommand{\LaTeX}{\textrm{\Oldlatex}}
    % Document parameters
    % Document title
    \title{build\_cnn}
    
    
    
    
    
    
    
% Pygments definitions
\makeatletter
\def\PY@reset{\let\PY@it=\relax \let\PY@bf=\relax%
    \let\PY@ul=\relax \let\PY@tc=\relax%
    \let\PY@bc=\relax \let\PY@ff=\relax}
\def\PY@tok#1{\csname PY@tok@#1\endcsname}
\def\PY@toks#1+{\ifx\relax#1\empty\else%
    \PY@tok{#1}\expandafter\PY@toks\fi}
\def\PY@do#1{\PY@bc{\PY@tc{\PY@ul{%
    \PY@it{\PY@bf{\PY@ff{#1}}}}}}}
\def\PY#1#2{\PY@reset\PY@toks#1+\relax+\PY@do{#2}}

\@namedef{PY@tok@w}{\def\PY@tc##1{\textcolor[rgb]{0.73,0.73,0.73}{##1}}}
\@namedef{PY@tok@c}{\let\PY@it=\textit\def\PY@tc##1{\textcolor[rgb]{0.24,0.48,0.48}{##1}}}
\@namedef{PY@tok@cp}{\def\PY@tc##1{\textcolor[rgb]{0.61,0.40,0.00}{##1}}}
\@namedef{PY@tok@k}{\let\PY@bf=\textbf\def\PY@tc##1{\textcolor[rgb]{0.00,0.50,0.00}{##1}}}
\@namedef{PY@tok@kp}{\def\PY@tc##1{\textcolor[rgb]{0.00,0.50,0.00}{##1}}}
\@namedef{PY@tok@kt}{\def\PY@tc##1{\textcolor[rgb]{0.69,0.00,0.25}{##1}}}
\@namedef{PY@tok@o}{\def\PY@tc##1{\textcolor[rgb]{0.40,0.40,0.40}{##1}}}
\@namedef{PY@tok@ow}{\let\PY@bf=\textbf\def\PY@tc##1{\textcolor[rgb]{0.67,0.13,1.00}{##1}}}
\@namedef{PY@tok@nb}{\def\PY@tc##1{\textcolor[rgb]{0.00,0.50,0.00}{##1}}}
\@namedef{PY@tok@nf}{\def\PY@tc##1{\textcolor[rgb]{0.00,0.00,1.00}{##1}}}
\@namedef{PY@tok@nc}{\let\PY@bf=\textbf\def\PY@tc##1{\textcolor[rgb]{0.00,0.00,1.00}{##1}}}
\@namedef{PY@tok@nn}{\let\PY@bf=\textbf\def\PY@tc##1{\textcolor[rgb]{0.00,0.00,1.00}{##1}}}
\@namedef{PY@tok@ne}{\let\PY@bf=\textbf\def\PY@tc##1{\textcolor[rgb]{0.80,0.25,0.22}{##1}}}
\@namedef{PY@tok@nv}{\def\PY@tc##1{\textcolor[rgb]{0.10,0.09,0.49}{##1}}}
\@namedef{PY@tok@no}{\def\PY@tc##1{\textcolor[rgb]{0.53,0.00,0.00}{##1}}}
\@namedef{PY@tok@nl}{\def\PY@tc##1{\textcolor[rgb]{0.46,0.46,0.00}{##1}}}
\@namedef{PY@tok@ni}{\let\PY@bf=\textbf\def\PY@tc##1{\textcolor[rgb]{0.44,0.44,0.44}{##1}}}
\@namedef{PY@tok@na}{\def\PY@tc##1{\textcolor[rgb]{0.41,0.47,0.13}{##1}}}
\@namedef{PY@tok@nt}{\let\PY@bf=\textbf\def\PY@tc##1{\textcolor[rgb]{0.00,0.50,0.00}{##1}}}
\@namedef{PY@tok@nd}{\def\PY@tc##1{\textcolor[rgb]{0.67,0.13,1.00}{##1}}}
\@namedef{PY@tok@s}{\def\PY@tc##1{\textcolor[rgb]{0.73,0.13,0.13}{##1}}}
\@namedef{PY@tok@sd}{\let\PY@it=\textit\def\PY@tc##1{\textcolor[rgb]{0.73,0.13,0.13}{##1}}}
\@namedef{PY@tok@si}{\let\PY@bf=\textbf\def\PY@tc##1{\textcolor[rgb]{0.64,0.35,0.47}{##1}}}
\@namedef{PY@tok@se}{\let\PY@bf=\textbf\def\PY@tc##1{\textcolor[rgb]{0.67,0.36,0.12}{##1}}}
\@namedef{PY@tok@sr}{\def\PY@tc##1{\textcolor[rgb]{0.64,0.35,0.47}{##1}}}
\@namedef{PY@tok@ss}{\def\PY@tc##1{\textcolor[rgb]{0.10,0.09,0.49}{##1}}}
\@namedef{PY@tok@sx}{\def\PY@tc##1{\textcolor[rgb]{0.00,0.50,0.00}{##1}}}
\@namedef{PY@tok@m}{\def\PY@tc##1{\textcolor[rgb]{0.40,0.40,0.40}{##1}}}
\@namedef{PY@tok@gh}{\let\PY@bf=\textbf\def\PY@tc##1{\textcolor[rgb]{0.00,0.00,0.50}{##1}}}
\@namedef{PY@tok@gu}{\let\PY@bf=\textbf\def\PY@tc##1{\textcolor[rgb]{0.50,0.00,0.50}{##1}}}
\@namedef{PY@tok@gd}{\def\PY@tc##1{\textcolor[rgb]{0.63,0.00,0.00}{##1}}}
\@namedef{PY@tok@gi}{\def\PY@tc##1{\textcolor[rgb]{0.00,0.52,0.00}{##1}}}
\@namedef{PY@tok@gr}{\def\PY@tc##1{\textcolor[rgb]{0.89,0.00,0.00}{##1}}}
\@namedef{PY@tok@ge}{\let\PY@it=\textit}
\@namedef{PY@tok@gs}{\let\PY@bf=\textbf}
\@namedef{PY@tok@ges}{\let\PY@bf=\textbf\let\PY@it=\textit}
\@namedef{PY@tok@gp}{\let\PY@bf=\textbf\def\PY@tc##1{\textcolor[rgb]{0.00,0.00,0.50}{##1}}}
\@namedef{PY@tok@go}{\def\PY@tc##1{\textcolor[rgb]{0.44,0.44,0.44}{##1}}}
\@namedef{PY@tok@gt}{\def\PY@tc##1{\textcolor[rgb]{0.00,0.27,0.87}{##1}}}
\@namedef{PY@tok@err}{\def\PY@bc##1{{\setlength{\fboxsep}{\string -\fboxrule}\fcolorbox[rgb]{1.00,0.00,0.00}{1,1,1}{\strut ##1}}}}
\@namedef{PY@tok@kc}{\let\PY@bf=\textbf\def\PY@tc##1{\textcolor[rgb]{0.00,0.50,0.00}{##1}}}
\@namedef{PY@tok@kd}{\let\PY@bf=\textbf\def\PY@tc##1{\textcolor[rgb]{0.00,0.50,0.00}{##1}}}
\@namedef{PY@tok@kn}{\let\PY@bf=\textbf\def\PY@tc##1{\textcolor[rgb]{0.00,0.50,0.00}{##1}}}
\@namedef{PY@tok@kr}{\let\PY@bf=\textbf\def\PY@tc##1{\textcolor[rgb]{0.00,0.50,0.00}{##1}}}
\@namedef{PY@tok@bp}{\def\PY@tc##1{\textcolor[rgb]{0.00,0.50,0.00}{##1}}}
\@namedef{PY@tok@fm}{\def\PY@tc##1{\textcolor[rgb]{0.00,0.00,1.00}{##1}}}
\@namedef{PY@tok@vc}{\def\PY@tc##1{\textcolor[rgb]{0.10,0.09,0.49}{##1}}}
\@namedef{PY@tok@vg}{\def\PY@tc##1{\textcolor[rgb]{0.10,0.09,0.49}{##1}}}
\@namedef{PY@tok@vi}{\def\PY@tc##1{\textcolor[rgb]{0.10,0.09,0.49}{##1}}}
\@namedef{PY@tok@vm}{\def\PY@tc##1{\textcolor[rgb]{0.10,0.09,0.49}{##1}}}
\@namedef{PY@tok@sa}{\def\PY@tc##1{\textcolor[rgb]{0.73,0.13,0.13}{##1}}}
\@namedef{PY@tok@sb}{\def\PY@tc##1{\textcolor[rgb]{0.73,0.13,0.13}{##1}}}
\@namedef{PY@tok@sc}{\def\PY@tc##1{\textcolor[rgb]{0.73,0.13,0.13}{##1}}}
\@namedef{PY@tok@dl}{\def\PY@tc##1{\textcolor[rgb]{0.73,0.13,0.13}{##1}}}
\@namedef{PY@tok@s2}{\def\PY@tc##1{\textcolor[rgb]{0.73,0.13,0.13}{##1}}}
\@namedef{PY@tok@sh}{\def\PY@tc##1{\textcolor[rgb]{0.73,0.13,0.13}{##1}}}
\@namedef{PY@tok@s1}{\def\PY@tc##1{\textcolor[rgb]{0.73,0.13,0.13}{##1}}}
\@namedef{PY@tok@mb}{\def\PY@tc##1{\textcolor[rgb]{0.40,0.40,0.40}{##1}}}
\@namedef{PY@tok@mf}{\def\PY@tc##1{\textcolor[rgb]{0.40,0.40,0.40}{##1}}}
\@namedef{PY@tok@mh}{\def\PY@tc##1{\textcolor[rgb]{0.40,0.40,0.40}{##1}}}
\@namedef{PY@tok@mi}{\def\PY@tc##1{\textcolor[rgb]{0.40,0.40,0.40}{##1}}}
\@namedef{PY@tok@il}{\def\PY@tc##1{\textcolor[rgb]{0.40,0.40,0.40}{##1}}}
\@namedef{PY@tok@mo}{\def\PY@tc##1{\textcolor[rgb]{0.40,0.40,0.40}{##1}}}
\@namedef{PY@tok@ch}{\let\PY@it=\textit\def\PY@tc##1{\textcolor[rgb]{0.24,0.48,0.48}{##1}}}
\@namedef{PY@tok@cm}{\let\PY@it=\textit\def\PY@tc##1{\textcolor[rgb]{0.24,0.48,0.48}{##1}}}
\@namedef{PY@tok@cpf}{\let\PY@it=\textit\def\PY@tc##1{\textcolor[rgb]{0.24,0.48,0.48}{##1}}}
\@namedef{PY@tok@c1}{\let\PY@it=\textit\def\PY@tc##1{\textcolor[rgb]{0.24,0.48,0.48}{##1}}}
\@namedef{PY@tok@cs}{\let\PY@it=\textit\def\PY@tc##1{\textcolor[rgb]{0.24,0.48,0.48}{##1}}}

\def\PYZbs{\char`\\}
\def\PYZus{\char`\_}
\def\PYZob{\char`\{}
\def\PYZcb{\char`\}}
\def\PYZca{\char`\^}
\def\PYZam{\char`\&}
\def\PYZlt{\char`\<}
\def\PYZgt{\char`\>}
\def\PYZsh{\char`\#}
\def\PYZpc{\char`\%}
\def\PYZdl{\char`\$}
\def\PYZhy{\char`\-}
\def\PYZsq{\char`\'}
\def\PYZdq{\char`\"}
\def\PYZti{\char`\~}
% for compatibility with earlier versions
\def\PYZat{@}
\def\PYZlb{[}
\def\PYZrb{]}
\makeatother


    % For linebreaks inside Verbatim environment from package fancyvrb.
    \makeatletter
        \newbox\Wrappedcontinuationbox
        \newbox\Wrappedvisiblespacebox
        \newcommand*\Wrappedvisiblespace {\textcolor{red}{\textvisiblespace}}
        \newcommand*\Wrappedcontinuationsymbol {\textcolor{red}{\llap{\tiny$\m@th\hookrightarrow$}}}
        \newcommand*\Wrappedcontinuationindent {3ex }
        \newcommand*\Wrappedafterbreak {\kern\Wrappedcontinuationindent\copy\Wrappedcontinuationbox}
        % Take advantage of the already applied Pygments mark-up to insert
        % potential linebreaks for TeX processing.
        %        {, <, #, %, $, ' and ": go to next line.
        %        _, }, ^, &, >, - and ~: stay at end of broken line.
        % Use of \textquotesingle for straight quote.
        \newcommand*\Wrappedbreaksatspecials {%
            \def\PYGZus{\discretionary{\char`\_}{\Wrappedafterbreak}{\char`\_}}%
            \def\PYGZob{\discretionary{}{\Wrappedafterbreak\char`\{}{\char`\{}}%
            \def\PYGZcb{\discretionary{\char`\}}{\Wrappedafterbreak}{\char`\}}}%
            \def\PYGZca{\discretionary{\char`\^}{\Wrappedafterbreak}{\char`\^}}%
            \def\PYGZam{\discretionary{\char`\&}{\Wrappedafterbreak}{\char`\&}}%
            \def\PYGZlt{\discretionary{}{\Wrappedafterbreak\char`\<}{\char`\<}}%
            \def\PYGZgt{\discretionary{\char`\>}{\Wrappedafterbreak}{\char`\>}}%
            \def\PYGZsh{\discretionary{}{\Wrappedafterbreak\char`\#}{\char`\#}}%
            \def\PYGZpc{\discretionary{}{\Wrappedafterbreak\char`\%}{\char`\%}}%
            \def\PYGZdl{\discretionary{}{\Wrappedafterbreak\char`\$}{\char`\$}}%
            \def\PYGZhy{\discretionary{\char`\-}{\Wrappedafterbreak}{\char`\-}}%
            \def\PYGZsq{\discretionary{}{\Wrappedafterbreak\textquotesingle}{\textquotesingle}}%
            \def\PYGZdq{\discretionary{}{\Wrappedafterbreak\char`\"}{\char`\"}}%
            \def\PYGZti{\discretionary{\char`\~}{\Wrappedafterbreak}{\char`\~}}%
        }
        % Some characters . , ; ? ! / are not pygmentized.
        % This macro makes them "active" and they will insert potential linebreaks
        \newcommand*\Wrappedbreaksatpunct {%
            \lccode`\~`\.\lowercase{\def~}{\discretionary{\hbox{\char`\.}}{\Wrappedafterbreak}{\hbox{\char`\.}}}%
            \lccode`\~`\,\lowercase{\def~}{\discretionary{\hbox{\char`\,}}{\Wrappedafterbreak}{\hbox{\char`\,}}}%
            \lccode`\~`\;\lowercase{\def~}{\discretionary{\hbox{\char`\;}}{\Wrappedafterbreak}{\hbox{\char`\;}}}%
            \lccode`\~`\:\lowercase{\def~}{\discretionary{\hbox{\char`\:}}{\Wrappedafterbreak}{\hbox{\char`\:}}}%
            \lccode`\~`\?\lowercase{\def~}{\discretionary{\hbox{\char`\?}}{\Wrappedafterbreak}{\hbox{\char`\?}}}%
            \lccode`\~`\!\lowercase{\def~}{\discretionary{\hbox{\char`\!}}{\Wrappedafterbreak}{\hbox{\char`\!}}}%
            \lccode`\~`\/\lowercase{\def~}{\discretionary{\hbox{\char`\/}}{\Wrappedafterbreak}{\hbox{\char`\/}}}%
            \catcode`\.\active
            \catcode`\,\active
            \catcode`\;\active
            \catcode`\:\active
            \catcode`\?\active
            \catcode`\!\active
            \catcode`\/\active
            \lccode`\~`\~
        }
    \makeatother

    \let\OriginalVerbatim=\Verbatim
    \makeatletter
    \renewcommand{\Verbatim}[1][1]{%
        %\parskip\z@skip
        \sbox\Wrappedcontinuationbox {\Wrappedcontinuationsymbol}%
        \sbox\Wrappedvisiblespacebox {\FV@SetupFont\Wrappedvisiblespace}%
        \def\FancyVerbFormatLine ##1{\hsize\linewidth
            \vtop{\raggedright\hyphenpenalty\z@\exhyphenpenalty\z@
                \doublehyphendemerits\z@\finalhyphendemerits\z@
                \strut ##1\strut}%
        }%
        % If the linebreak is at a space, the latter will be displayed as visible
        % space at end of first line, and a continuation symbol starts next line.
        % Stretch/shrink are however usually zero for typewriter font.
        \def\FV@Space {%
            \nobreak\hskip\z@ plus\fontdimen3\font minus\fontdimen4\font
            \discretionary{\copy\Wrappedvisiblespacebox}{\Wrappedafterbreak}
            {\kern\fontdimen2\font}%
        }%

        % Allow breaks at special characters using \PYG... macros.
        \Wrappedbreaksatspecials
        % Breaks at punctuation characters . , ; ? ! and / need catcode=\active
        \OriginalVerbatim[#1,codes*=\Wrappedbreaksatpunct]%
    }
    \makeatother

    % Exact colors from NB
    \definecolor{incolor}{HTML}{303F9F}
    \definecolor{outcolor}{HTML}{D84315}
    \definecolor{cellborder}{HTML}{CFCFCF}
    \definecolor{cellbackground}{HTML}{F7F7F7}

    % prompt
    \makeatletter
    \newcommand{\boxspacing}{\kern\kvtcb@left@rule\kern\kvtcb@boxsep}
    \makeatother
    \newcommand{\prompt}[4]{
        {\ttfamily\llap{{\color{#2}[#3]:\hspace{3pt}#4}}\vspace{-\baselineskip}}
    }
    

    
    % Prevent overflowing lines due to hard-to-break entities
    \sloppy
    % Setup hyperref package
    \hypersetup{
      breaklinks=true,  % so long urls are correctly broken across lines
      colorlinks=true,
      urlcolor=urlcolor,
      linkcolor=linkcolor,
      citecolor=citecolor,
      }
    % Slightly bigger margins than the latex defaults
    
    \geometry{verbose,tmargin=1in,bmargin=1in,lmargin=1in,rmargin=1in}
    
    

\begin{document}
    
    \maketitle
    
    

    
    \subsection{\# Convolutional Neural Networks - Build
Model}\label{convolutional-neural-networks---build-model}

In this notebook, we build and train a \textbf{CNN} to classify images
from the CIFAR-10 database. * The code provided here are \textbf{almost}
working. You are required to build up a CNN model and train it. * Make
sure you covered implementations of the \textbf{TODO}s in this notebook

The images in this database are small color images that fall into one of
ten classes; some example images are pictured below.

    \subsubsection{\texorpdfstring{Optional: Use
\href{http://pytorch.org/docs/stable/cuda.html}{CUDA} if
Available}{Optional: Use CUDA if Available}}\label{optional-use-cuda-if-available}

Since these are color (32x32x3) images, it may prove useful to speed up
your training time by using a GPU. CUDA is a parallel computing platform
and CUDA Tensors are the same as typical Tensors, but they utilize GPU's
for effcient parallel computation.

    \begin{tcolorbox}[breakable, size=fbox, boxrule=1pt, pad at break*=1mm,colback=cellbackground, colframe=cellborder]
\prompt{In}{incolor}{1}{\boxspacing}
\begin{Verbatim}[commandchars=\\\{\}]
\PY{k+kn}{import}\PY{+w}{ }\PY{n+nn}{torch}
\PY{k+kn}{import}\PY{+w}{ }\PY{n+nn}{numpy}\PY{+w}{ }\PY{k}{as}\PY{+w}{ }\PY{n+nn}{np}

\PY{c+c1}{\PYZsh{} check if CUDA is available}
\PY{n}{train\PYZus{}on\PYZus{}gpu} \PY{o}{=} \PY{n}{torch}\PY{o}{.}\PY{n}{cuda}\PY{o}{.}\PY{n}{is\PYZus{}available}\PY{p}{(}\PY{p}{)}

\PY{k}{if} \PY{o+ow}{not} \PY{n}{train\PYZus{}on\PYZus{}gpu}\PY{p}{:}
    \PY{n+nb}{print}\PY{p}{(}\PY{l+s+s1}{\PYZsq{}}\PY{l+s+s1}{CUDA is not available.  Training on CPU ...}\PY{l+s+s1}{\PYZsq{}}\PY{p}{)}
\PY{k}{else}\PY{p}{:}
    \PY{n+nb}{print}\PY{p}{(}\PY{l+s+s1}{\PYZsq{}}\PY{l+s+s1}{CUDA is available!  Training on GPU ...}\PY{l+s+s1}{\PYZsq{}}\PY{p}{)}
\end{Verbatim}
\end{tcolorbox}

    \begin{Verbatim}[commandchars=\\\{\}]
CUDA is not available.  Training on CPU {\ldots}
    \end{Verbatim}

    \subsection{\texorpdfstring{\#\# Load the
\href{http://pytorch.org/docs/stable/torchvision/datasets.html}{Data}}{\#\# Load the Data}}\label{load-the-data}

Downloading may take a minute. We load in the training and test data,
split the training data into a training and validation set, then create
DataLoaders for each of these sets of data.

    \begin{tcolorbox}[breakable, size=fbox, boxrule=1pt, pad at break*=1mm,colback=cellbackground, colframe=cellborder]
\prompt{In}{incolor}{2}{\boxspacing}
\begin{Verbatim}[commandchars=\\\{\}]
\PY{k+kn}{from}\PY{+w}{ }\PY{n+nn}{torchvision}\PY{+w}{ }\PY{k+kn}{import} \PY{n}{datasets}
\PY{k+kn}{import}\PY{+w}{ }\PY{n+nn}{torchvision}\PY{n+nn}{.}\PY{n+nn}{transforms}\PY{+w}{ }\PY{k}{as}\PY{+w}{ }\PY{n+nn}{transforms}
\PY{k+kn}{from}\PY{+w}{ }\PY{n+nn}{torch}\PY{n+nn}{.}\PY{n+nn}{utils}\PY{n+nn}{.}\PY{n+nn}{data}\PY{n+nn}{.}\PY{n+nn}{sampler}\PY{+w}{ }\PY{k+kn}{import} \PY{n}{SubsetRandomSampler}

\PY{c+c1}{\PYZsh{} number of subprocesses to use for data loading}
\PY{n}{num\PYZus{}workers} \PY{o}{=} \PY{l+m+mi}{0}
\PY{c+c1}{\PYZsh{} how many samples per batch to load}
\PY{n}{batch\PYZus{}size} \PY{o}{=} \PY{l+m+mi}{20}
\PY{c+c1}{\PYZsh{} percentage of training set to use as validation}
\PY{n}{valid\PYZus{}size} \PY{o}{=} \PY{l+m+mf}{0.2}

\PY{c+c1}{\PYZsh{} convert data to a normalized torch.FloatTensor}
\PY{n}{transform} \PY{o}{=} \PY{n}{transforms}\PY{o}{.}\PY{n}{Compose}\PY{p}{(}\PY{p}{[}
    \PY{n}{transforms}\PY{o}{.}\PY{n}{ToTensor}\PY{p}{(}\PY{p}{)}\PY{p}{,}
    \PY{n}{transforms}\PY{o}{.}\PY{n}{Normalize}\PY{p}{(}\PY{p}{(}\PY{l+m+mf}{0.5}\PY{p}{,} \PY{l+m+mf}{0.5}\PY{p}{,} \PY{l+m+mf}{0.5}\PY{p}{)}\PY{p}{,} \PY{p}{(}\PY{l+m+mf}{0.5}\PY{p}{,} \PY{l+m+mf}{0.5}\PY{p}{,} \PY{l+m+mf}{0.5}\PY{p}{)}\PY{p}{)}
    \PY{p}{]}\PY{p}{)}

\PY{c+c1}{\PYZsh{} choose the training and test datasets}
\PY{n}{train\PYZus{}data} \PY{o}{=} \PY{n}{datasets}\PY{o}{.}\PY{n}{CIFAR10}\PY{p}{(}\PY{l+s+s1}{\PYZsq{}}\PY{l+s+s1}{data}\PY{l+s+s1}{\PYZsq{}}\PY{p}{,} \PY{n}{train}\PY{o}{=}\PY{k+kc}{True}\PY{p}{,}
                              \PY{n}{download}\PY{o}{=}\PY{k+kc}{True}\PY{p}{,} \PY{n}{transform}\PY{o}{=}\PY{n}{transform}\PY{p}{)}
\PY{n}{test\PYZus{}data} \PY{o}{=} \PY{n}{datasets}\PY{o}{.}\PY{n}{CIFAR10}\PY{p}{(}\PY{l+s+s1}{\PYZsq{}}\PY{l+s+s1}{data}\PY{l+s+s1}{\PYZsq{}}\PY{p}{,} \PY{n}{train}\PY{o}{=}\PY{k+kc}{False}\PY{p}{,}
                             \PY{n}{download}\PY{o}{=}\PY{k+kc}{True}\PY{p}{,} \PY{n}{transform}\PY{o}{=}\PY{n}{transform}\PY{p}{)}

\PY{c+c1}{\PYZsh{} obtain training indices that will be used for validation}
\PY{n}{num\PYZus{}train} \PY{o}{=} \PY{n+nb}{len}\PY{p}{(}\PY{n}{train\PYZus{}data}\PY{p}{)}
\PY{n}{indices} \PY{o}{=} \PY{n+nb}{list}\PY{p}{(}\PY{n+nb}{range}\PY{p}{(}\PY{n}{num\PYZus{}train}\PY{p}{)}\PY{p}{)}
\PY{n}{np}\PY{o}{.}\PY{n}{random}\PY{o}{.}\PY{n}{shuffle}\PY{p}{(}\PY{n}{indices}\PY{p}{)}
\PY{n}{split} \PY{o}{=} \PY{n+nb}{int}\PY{p}{(}\PY{n}{np}\PY{o}{.}\PY{n}{floor}\PY{p}{(}\PY{n}{valid\PYZus{}size} \PY{o}{*} \PY{n}{num\PYZus{}train}\PY{p}{)}\PY{p}{)}
\PY{n}{train\PYZus{}idx}\PY{p}{,} \PY{n}{valid\PYZus{}idx} \PY{o}{=} \PY{n}{indices}\PY{p}{[}\PY{n}{split}\PY{p}{:}\PY{p}{]}\PY{p}{,} \PY{n}{indices}\PY{p}{[}\PY{p}{:}\PY{n}{split}\PY{p}{]}

\PY{c+c1}{\PYZsh{} define samplers for obtaining training and validation batches}
\PY{n}{train\PYZus{}sampler} \PY{o}{=} \PY{n}{SubsetRandomSampler}\PY{p}{(}\PY{n}{train\PYZus{}idx}\PY{p}{)}
\PY{n}{valid\PYZus{}sampler} \PY{o}{=} \PY{n}{SubsetRandomSampler}\PY{p}{(}\PY{n}{valid\PYZus{}idx}\PY{p}{)}

\PY{c+c1}{\PYZsh{} prepare data loaders (combine dataset and sampler)}
\PY{n}{train\PYZus{}loader} \PY{o}{=} \PY{n}{torch}\PY{o}{.}\PY{n}{utils}\PY{o}{.}\PY{n}{data}\PY{o}{.}\PY{n}{DataLoader}\PY{p}{(}\PY{n}{train\PYZus{}data}\PY{p}{,} \PY{n}{batch\PYZus{}size}\PY{o}{=}\PY{n}{batch\PYZus{}size}\PY{p}{,}
    \PY{n}{sampler}\PY{o}{=}\PY{n}{train\PYZus{}sampler}\PY{p}{,} \PY{n}{num\PYZus{}workers}\PY{o}{=}\PY{n}{num\PYZus{}workers}\PY{p}{)}
\PY{n}{valid\PYZus{}loader} \PY{o}{=} \PY{n}{torch}\PY{o}{.}\PY{n}{utils}\PY{o}{.}\PY{n}{data}\PY{o}{.}\PY{n}{DataLoader}\PY{p}{(}\PY{n}{train\PYZus{}data}\PY{p}{,} \PY{n}{batch\PYZus{}size}\PY{o}{=}\PY{n}{batch\PYZus{}size}\PY{p}{,} 
    \PY{n}{sampler}\PY{o}{=}\PY{n}{valid\PYZus{}sampler}\PY{p}{,} \PY{n}{num\PYZus{}workers}\PY{o}{=}\PY{n}{num\PYZus{}workers}\PY{p}{)}
\PY{n}{test\PYZus{}loader} \PY{o}{=} \PY{n}{torch}\PY{o}{.}\PY{n}{utils}\PY{o}{.}\PY{n}{data}\PY{o}{.}\PY{n}{DataLoader}\PY{p}{(}\PY{n}{test\PYZus{}data}\PY{p}{,} \PY{n}{batch\PYZus{}size}\PY{o}{=}\PY{n}{batch\PYZus{}size}\PY{p}{,} 
    \PY{n}{num\PYZus{}workers}\PY{o}{=}\PY{n}{num\PYZus{}workers}\PY{p}{)}

\PY{c+c1}{\PYZsh{} specify the image classes}
\PY{n}{classes} \PY{o}{=} \PY{p}{[}\PY{l+s+s1}{\PYZsq{}}\PY{l+s+s1}{airplane}\PY{l+s+s1}{\PYZsq{}}\PY{p}{,} \PY{l+s+s1}{\PYZsq{}}\PY{l+s+s1}{automobile}\PY{l+s+s1}{\PYZsq{}}\PY{p}{,} \PY{l+s+s1}{\PYZsq{}}\PY{l+s+s1}{bird}\PY{l+s+s1}{\PYZsq{}}\PY{p}{,} \PY{l+s+s1}{\PYZsq{}}\PY{l+s+s1}{cat}\PY{l+s+s1}{\PYZsq{}}\PY{p}{,} \PY{l+s+s1}{\PYZsq{}}\PY{l+s+s1}{deer}\PY{l+s+s1}{\PYZsq{}}\PY{p}{,}
           \PY{l+s+s1}{\PYZsq{}}\PY{l+s+s1}{dog}\PY{l+s+s1}{\PYZsq{}}\PY{p}{,} \PY{l+s+s1}{\PYZsq{}}\PY{l+s+s1}{frog}\PY{l+s+s1}{\PYZsq{}}\PY{p}{,} \PY{l+s+s1}{\PYZsq{}}\PY{l+s+s1}{horse}\PY{l+s+s1}{\PYZsq{}}\PY{p}{,} \PY{l+s+s1}{\PYZsq{}}\PY{l+s+s1}{ship}\PY{l+s+s1}{\PYZsq{}}\PY{p}{,} \PY{l+s+s1}{\PYZsq{}}\PY{l+s+s1}{truck}\PY{l+s+s1}{\PYZsq{}}\PY{p}{]}
\end{Verbatim}
\end{tcolorbox}

    \subsubsection{Visualize a Batch of Training
Data}\label{visualize-a-batch-of-training-data}

    \begin{tcolorbox}[breakable, size=fbox, boxrule=1pt, pad at break*=1mm,colback=cellbackground, colframe=cellborder]
\prompt{In}{incolor}{3}{\boxspacing}
\begin{Verbatim}[commandchars=\\\{\}]
\PY{k+kn}{import}\PY{+w}{ }\PY{n+nn}{matplotlib}\PY{n+nn}{.}\PY{n+nn}{pyplot}\PY{+w}{ }\PY{k}{as}\PY{+w}{ }\PY{n+nn}{plt}
\PY{o}{\PYZpc{}}\PY{k}{matplotlib} inline

\PY{c+c1}{\PYZsh{} helper function to un\PYZhy{}normalize and display an image}
\PY{k}{def}\PY{+w}{ }\PY{n+nf}{imshow}\PY{p}{(}\PY{n}{img}\PY{p}{)}\PY{p}{:}
    \PY{n}{img} \PY{o}{=} \PY{n}{img} \PY{o}{/} \PY{l+m+mi}{2} \PY{o}{+} \PY{l+m+mf}{0.5}  \PY{c+c1}{\PYZsh{} unnormalize}
    \PY{n}{plt}\PY{o}{.}\PY{n}{imshow}\PY{p}{(}\PY{n}{np}\PY{o}{.}\PY{n}{transpose}\PY{p}{(}\PY{n}{img}\PY{p}{,} \PY{p}{(}\PY{l+m+mi}{1}\PY{p}{,} \PY{l+m+mi}{2}\PY{p}{,} \PY{l+m+mi}{0}\PY{p}{)}\PY{p}{)}\PY{p}{)}  \PY{c+c1}{\PYZsh{} convert from Tensor image}
\end{Verbatim}
\end{tcolorbox}

    \begin{tcolorbox}[breakable, size=fbox, boxrule=1pt, pad at break*=1mm,colback=cellbackground, colframe=cellborder]
\prompt{In}{incolor}{4}{\boxspacing}
\begin{Verbatim}[commandchars=\\\{\}]
\PY{c+c1}{\PYZsh{} obtain one batch of training images}
\PY{n}{dataiter} \PY{o}{=} \PY{n+nb}{iter}\PY{p}{(}\PY{n}{train\PYZus{}loader}\PY{p}{)}
\PY{c+c1}{\PYZsh{}images, labels = dataiter.next() \PYZsh{}python, torchvision version match issue}
\PY{n}{images}\PY{p}{,} \PY{n}{labels} \PY{o}{=} \PY{n+nb}{next}\PY{p}{(}\PY{n}{dataiter}\PY{p}{)}
\PY{n}{images} \PY{o}{=} \PY{n}{images}\PY{o}{.}\PY{n}{numpy}\PY{p}{(}\PY{p}{)} \PY{c+c1}{\PYZsh{} convert images to numpy for display}

\PY{c+c1}{\PYZsh{} plot the images in the batch, along with the corresponding labels}
\PY{n}{fig} \PY{o}{=} \PY{n}{plt}\PY{o}{.}\PY{n}{figure}\PY{p}{(}\PY{n}{figsize}\PY{o}{=}\PY{p}{(}\PY{l+m+mi}{25}\PY{p}{,} \PY{l+m+mi}{4}\PY{p}{)}\PY{p}{)}
\PY{c+c1}{\PYZsh{} display 20 images}
\PY{k}{for} \PY{n}{idx} \PY{o+ow}{in} \PY{n}{np}\PY{o}{.}\PY{n}{arange}\PY{p}{(}\PY{l+m+mi}{20}\PY{p}{)}\PY{p}{:}
    \PY{n}{ax} \PY{o}{=} \PY{n}{fig}\PY{o}{.}\PY{n}{add\PYZus{}subplot}\PY{p}{(}\PY{l+m+mi}{2}\PY{p}{,} \PY{n+nb}{int}\PY{p}{(}\PY{l+m+mi}{20}\PY{o}{/}\PY{l+m+mi}{2}\PY{p}{)}\PY{p}{,} \PY{n}{idx}\PY{o}{+}\PY{l+m+mi}{1}\PY{p}{,} \PY{n}{xticks}\PY{o}{=}\PY{p}{[}\PY{p}{]}\PY{p}{,} \PY{n}{yticks}\PY{o}{=}\PY{p}{[}\PY{p}{]}\PY{p}{)}
    \PY{n}{imshow}\PY{p}{(}\PY{n}{images}\PY{p}{[}\PY{n}{idx}\PY{p}{]}\PY{p}{)}
    \PY{n}{ax}\PY{o}{.}\PY{n}{set\PYZus{}title}\PY{p}{(}\PY{n}{classes}\PY{p}{[}\PY{n}{labels}\PY{p}{[}\PY{n}{idx}\PY{p}{]}\PY{p}{]}\PY{p}{)}
\end{Verbatim}
\end{tcolorbox}

    \begin{center}
    \adjustimage{max size={0.9\linewidth}{0.9\paperheight}}{output_7_0.png}
    \end{center}
    { \hspace*{\fill} \\}
    
    \subsubsection{View an Image in More
Detail}\label{view-an-image-in-more-detail}

Here, we look at the normalized red, green, and blue (RGB) color
channels as three separate, grayscale intensity images.

    \begin{tcolorbox}[breakable, size=fbox, boxrule=1pt, pad at break*=1mm,colback=cellbackground, colframe=cellborder]
\prompt{In}{incolor}{5}{\boxspacing}
\begin{Verbatim}[commandchars=\\\{\}]
\PY{n}{rgb\PYZus{}img} \PY{o}{=} \PY{n}{np}\PY{o}{.}\PY{n}{squeeze}\PY{p}{(}\PY{n}{images}\PY{p}{[}\PY{l+m+mi}{3}\PY{p}{]}\PY{p}{)}
\PY{n}{channels} \PY{o}{=} \PY{p}{[}\PY{l+s+s1}{\PYZsq{}}\PY{l+s+s1}{red channel}\PY{l+s+s1}{\PYZsq{}}\PY{p}{,} \PY{l+s+s1}{\PYZsq{}}\PY{l+s+s1}{green channel}\PY{l+s+s1}{\PYZsq{}}\PY{p}{,} \PY{l+s+s1}{\PYZsq{}}\PY{l+s+s1}{blue channel}\PY{l+s+s1}{\PYZsq{}}\PY{p}{]}

\PY{n}{fig} \PY{o}{=} \PY{n}{plt}\PY{o}{.}\PY{n}{figure}\PY{p}{(}\PY{n}{figsize} \PY{o}{=} \PY{p}{(}\PY{l+m+mi}{36}\PY{p}{,} \PY{l+m+mi}{36}\PY{p}{)}\PY{p}{)} 
\PY{k}{for} \PY{n}{idx} \PY{o+ow}{in} \PY{n}{np}\PY{o}{.}\PY{n}{arange}\PY{p}{(}\PY{n}{rgb\PYZus{}img}\PY{o}{.}\PY{n}{shape}\PY{p}{[}\PY{l+m+mi}{0}\PY{p}{]}\PY{p}{)}\PY{p}{:}
    \PY{n}{ax} \PY{o}{=} \PY{n}{fig}\PY{o}{.}\PY{n}{add\PYZus{}subplot}\PY{p}{(}\PY{l+m+mi}{1}\PY{p}{,} \PY{l+m+mi}{3}\PY{p}{,} \PY{n}{idx} \PY{o}{+} \PY{l+m+mi}{1}\PY{p}{)}
    \PY{n}{img} \PY{o}{=} \PY{n}{rgb\PYZus{}img}\PY{p}{[}\PY{n}{idx}\PY{p}{]}
    \PY{n}{ax}\PY{o}{.}\PY{n}{imshow}\PY{p}{(}\PY{n}{img}\PY{p}{,} \PY{n}{cmap}\PY{o}{=}\PY{l+s+s1}{\PYZsq{}}\PY{l+s+s1}{gray}\PY{l+s+s1}{\PYZsq{}}\PY{p}{)}
    \PY{n}{ax}\PY{o}{.}\PY{n}{set\PYZus{}title}\PY{p}{(}\PY{n}{channels}\PY{p}{[}\PY{n}{idx}\PY{p}{]}\PY{p}{)}
    \PY{n}{width}\PY{p}{,} \PY{n}{height} \PY{o}{=} \PY{n}{img}\PY{o}{.}\PY{n}{shape}
    \PY{n}{thresh} \PY{o}{=} \PY{n}{img}\PY{o}{.}\PY{n}{max}\PY{p}{(}\PY{p}{)}\PY{o}{/}\PY{l+m+mf}{2.5}
    \PY{k}{for} \PY{n}{x} \PY{o+ow}{in} \PY{n+nb}{range}\PY{p}{(}\PY{n}{width}\PY{p}{)}\PY{p}{:}
        \PY{k}{for} \PY{n}{y} \PY{o+ow}{in} \PY{n+nb}{range}\PY{p}{(}\PY{n}{height}\PY{p}{)}\PY{p}{:}
            \PY{n}{val} \PY{o}{=} \PY{n+nb}{round}\PY{p}{(}\PY{n}{img}\PY{p}{[}\PY{n}{x}\PY{p}{]}\PY{p}{[}\PY{n}{y}\PY{p}{]}\PY{p}{,}\PY{l+m+mi}{2}\PY{p}{)} \PY{k}{if} \PY{n}{img}\PY{p}{[}\PY{n}{x}\PY{p}{]}\PY{p}{[}\PY{n}{y}\PY{p}{]} \PY{o}{!=}\PY{l+m+mi}{0} \PY{k}{else} \PY{l+m+mi}{0}
            \PY{n}{ax}\PY{o}{.}\PY{n}{annotate}\PY{p}{(}\PY{n+nb}{str}\PY{p}{(}\PY{n}{val}\PY{p}{)}\PY{p}{,} \PY{n}{xy}\PY{o}{=}\PY{p}{(}\PY{n}{y}\PY{p}{,}\PY{n}{x}\PY{p}{)}\PY{p}{,}
                    \PY{n}{horizontalalignment}\PY{o}{=}\PY{l+s+s1}{\PYZsq{}}\PY{l+s+s1}{center}\PY{l+s+s1}{\PYZsq{}}\PY{p}{,}
                    \PY{n}{verticalalignment}\PY{o}{=}\PY{l+s+s1}{\PYZsq{}}\PY{l+s+s1}{center}\PY{l+s+s1}{\PYZsq{}}\PY{p}{,} \PY{n}{size}\PY{o}{=}\PY{l+m+mi}{8}\PY{p}{,}
                    \PY{n}{color}\PY{o}{=}\PY{l+s+s1}{\PYZsq{}}\PY{l+s+s1}{white}\PY{l+s+s1}{\PYZsq{}} \PY{k}{if} \PY{n}{img}\PY{p}{[}\PY{n}{x}\PY{p}{]}\PY{p}{[}\PY{n}{y}\PY{p}{]}\PY{o}{\PYZlt{}}\PY{n}{thresh} \PY{k}{else} \PY{l+s+s1}{\PYZsq{}}\PY{l+s+s1}{black}\PY{l+s+s1}{\PYZsq{}}\PY{p}{)}
\end{Verbatim}
\end{tcolorbox}

    \begin{center}
    \adjustimage{max size={0.9\linewidth}{0.9\paperheight}}{output_9_0.png}
    \end{center}
    { \hspace*{\fill} \\}
    
    \subsection{\texorpdfstring{\#\# \textbf{TODO}: Define the Network
\href{http://pytorch.org/docs/stable/nn.html}{Architecture}}{\#\# TODO: Define the Network Architecture}}\label{todo-define-the-network-architecture}

Build up your own Convolutional Neural Network using Pytorch API: -
nn.Conv2d(): for convolution - nn.MaxPool2d(): for maxpooling (spatial
resolution reduction) - nn.Linear(): for last 1 or 2 layers of fully
connected layer before the output layer. - nn.Dropout(): optional,
\href{https://pytorch.org/docs/stable/generated/torch.nn.Dropout.html}{dropout}
can be used to avoid overfitting. - F.relu(): Use ReLU as the activation
function for all the hidden layers

The following is a skeleton example that's not completely working.

    \begin{tcolorbox}[breakable, size=fbox, boxrule=1pt, pad at break*=1mm,colback=cellbackground, colframe=cellborder]
\prompt{In}{incolor}{6}{\boxspacing}
\begin{Verbatim}[commandchars=\\\{\}]
\PY{k+kn}{import}\PY{+w}{ }\PY{n+nn}{torch}\PY{n+nn}{.}\PY{n+nn}{nn}\PY{+w}{ }\PY{k}{as}\PY{+w}{ }\PY{n+nn}{nn}
\PY{k+kn}{import}\PY{+w}{ }\PY{n+nn}{torch}\PY{n+nn}{.}\PY{n+nn}{nn}\PY{n+nn}{.}\PY{n+nn}{functional}\PY{+w}{ }\PY{k}{as}\PY{+w}{ }\PY{n+nn}{F}

\PY{c+c1}{\PYZsh{} define the CNN architecture}
\PY{k}{class}\PY{+w}{ }\PY{n+nc}{Net}\PY{p}{(}\PY{n}{nn}\PY{o}{.}\PY{n}{Module}\PY{p}{)}\PY{p}{:}
    \PY{k}{def}\PY{+w}{ }\PY{n+nf+fm}{\PYZus{}\PYZus{}init\PYZus{}\PYZus{}}\PY{p}{(}\PY{n+nb+bp}{self}\PY{p}{)}\PY{p}{:}
        \PY{n+nb}{super}\PY{p}{(}\PY{n}{Net}\PY{p}{,} \PY{n+nb+bp}{self}\PY{p}{)}\PY{o}{.}\PY{n+nf+fm}{\PYZus{}\PYZus{}init\PYZus{}\PYZus{}}\PY{p}{(}\PY{p}{)}
        
        \PY{n+nb+bp}{self}\PY{o}{.}\PY{n}{conv1} \PY{o}{=} \PY{n}{nn}\PY{o}{.}\PY{n}{Conv2d}\PY{p}{(}\PY{l+m+mi}{3}\PY{p}{,} \PY{l+m+mi}{16}\PY{p}{,} \PY{l+m+mi}{3}\PY{p}{,} \PY{n}{padding}\PY{o}{=}\PY{l+m+mi}{1}\PY{p}{)}
        \PY{n+nb+bp}{self}\PY{o}{.}\PY{n}{conv2} \PY{o}{=} \PY{n}{nn}\PY{o}{.}\PY{n}{Conv2d}\PY{p}{(}\PY{l+m+mi}{16}\PY{p}{,} \PY{l+m+mi}{32}\PY{p}{,} \PY{l+m+mi}{3}\PY{p}{,} \PY{n}{padding}\PY{o}{=}\PY{l+m+mi}{1}\PY{p}{)}
        
        \PY{c+c1}{\PYZsh{} max pooling layer}
        \PY{n+nb+bp}{self}\PY{o}{.}\PY{n}{pool} \PY{o}{=} \PY{n}{nn}\PY{o}{.}\PY{n}{MaxPool2d}\PY{p}{(}\PY{l+m+mi}{2}\PY{p}{,} \PY{l+m+mi}{2}\PY{p}{)}
        
        \PY{n+nb+bp}{self}\PY{o}{.}\PY{n}{fc1} \PY{o}{=} \PY{n}{nn}\PY{o}{.}\PY{n}{Linear}\PY{p}{(}\PY{l+m+mi}{32} \PY{o}{*} \PY{l+m+mi}{8} \PY{o}{*} \PY{l+m+mi}{8}\PY{p}{,} \PY{l+m+mi}{500}\PY{p}{)}
        \PY{n+nb+bp}{self}\PY{o}{.}\PY{n}{fc2} \PY{o}{=} \PY{n}{nn}\PY{o}{.}\PY{n}{Linear}\PY{p}{(}\PY{l+m+mi}{500}\PY{p}{,} \PY{l+m+mi}{10}\PY{p}{)}

        \PY{n+nb+bp}{self}\PY{o}{.}\PY{n}{dropout} \PY{o}{=} \PY{n}{nn}\PY{o}{.}\PY{n}{Dropout}\PY{p}{(}\PY{l+m+mf}{0.25}\PY{p}{)}
        \PY{c+c1}{\PYZsh{} example self.dropout = nn.Dropout(0.25)}

    \PY{k}{def}\PY{+w}{ }\PY{n+nf}{forward}\PY{p}{(}\PY{n+nb+bp}{self}\PY{p}{,} \PY{n}{x}\PY{p}{)}\PY{p}{:}
        \PY{c+c1}{\PYZsh{} add sequence of convolutional and max pooling layers}
        \PY{c+c1}{\PYZsh{} assume we have 2 convolutional layers defined agove}
        \PY{c+c1}{\PYZsh{} and we do a maxpooling after each conv layer}
        \PY{n}{x} \PY{o}{=} \PY{n+nb+bp}{self}\PY{o}{.}\PY{n}{pool}\PY{p}{(}\PY{n}{F}\PY{o}{.}\PY{n}{relu}\PY{p}{(}\PY{n+nb+bp}{self}\PY{o}{.}\PY{n}{conv1}\PY{p}{(}\PY{n}{x}\PY{p}{)}\PY{p}{)}\PY{p}{)}
        \PY{n}{x} \PY{o}{=} \PY{n+nb+bp}{self}\PY{o}{.}\PY{n}{pool}\PY{p}{(}\PY{n}{F}\PY{o}{.}\PY{n}{relu}\PY{p}{(}\PY{n+nb+bp}{self}\PY{o}{.}\PY{n}{conv2}\PY{p}{(}\PY{n}{x}\PY{p}{)}\PY{p}{)}\PY{p}{)}

        \PY{n}{x} \PY{o}{=} \PY{n}{x}\PY{o}{.}\PY{n}{view}\PY{p}{(}\PY{o}{\PYZhy{}}\PY{l+m+mi}{1}\PY{p}{,} \PY{l+m+mi}{32} \PY{o}{*} \PY{l+m+mi}{8} \PY{o}{*} \PY{l+m+mi}{8}\PY{p}{)}

        \PY{c+c1}{\PYZsh{} optional add dropout layer}
        \PY{c+c1}{\PYZsh{}x = self.dropout(x)}
        
        \PY{c+c1}{\PYZsh{} add 1st hidden layer, with relu activation function}
        \PY{n}{x} \PY{o}{=} \PY{n}{F}\PY{o}{.}\PY{n}{relu}\PY{p}{(}\PY{n+nb+bp}{self}\PY{o}{.}\PY{n}{fc1}\PY{p}{(}\PY{n}{x}\PY{p}{)}\PY{p}{)}
        \PY{c+c1}{\PYZsh{} optional add dropout layer}
        \PY{c+c1}{\PYZsh{}x = self.dropout(x)}
        \PY{c+c1}{\PYZsh{} add 2nd hidden layer, with relu activation function}
        \PY{n}{x} \PY{o}{=} \PY{n+nb+bp}{self}\PY{o}{.}\PY{n}{fc2}\PY{p}{(}\PY{n}{x}\PY{p}{)}
        \PY{k}{return} \PY{n}{x}

\PY{c+c1}{\PYZsh{} create a complete CNN}
\PY{n}{model} \PY{o}{=} \PY{n}{Net}\PY{p}{(}\PY{p}{)}
\PY{n+nb}{print}\PY{p}{(}\PY{n}{model}\PY{p}{)}

\PY{c+c1}{\PYZsh{} move tensors to GPU if CUDA is available}
\PY{k}{if} \PY{n}{train\PYZus{}on\PYZus{}gpu}\PY{p}{:}
    \PY{n}{model}\PY{o}{.}\PY{n}{cuda}\PY{p}{(}\PY{p}{)}
\end{Verbatim}
\end{tcolorbox}

    \begin{Verbatim}[commandchars=\\\{\}]
Net(
  (conv1): Conv2d(3, 16, kernel\_size=(3, 3), stride=(1, 1), padding=(1, 1))
  (conv2): Conv2d(16, 32, kernel\_size=(3, 3), stride=(1, 1), padding=(1, 1))
  (pool): MaxPool2d(kernel\_size=2, stride=2, padding=0, dilation=1,
ceil\_mode=False)
  (fc1): Linear(in\_features=2048, out\_features=500, bias=True)
  (fc2): Linear(in\_features=500, out\_features=10, bias=True)
  (dropout): Dropout(p=0.25, inplace=False)
)
    \end{Verbatim}

    \subsection{\texorpdfstring{\#\#\# Specify
\href{http://pytorch.org/docs/stable/nn.html\#loss-functions}{Loss
Function} and
\href{http://pytorch.org/docs/stable/optim.html}{Optimizer}}{\#\#\# Specify Loss Function and Optimizer}}\label{specify-loss-function-and-optimizer}

Decide on a loss and optimization function that is best suited for this
classification task. The linked code examples from above, may be a good
starting point;
\href{https://github.com/pytorch/tutorials/blob/master/beginner_source/blitz/cifar10_tutorial.py}{this
PyTorch classification example} Pay close attention to the value for
\textbf{learning rate} as this value determines how your model converges
to a small error.

The following is working code, but you can make your own adjustments.

\textbf{TODO}: try to compare with ADAM optimizer

    \begin{tcolorbox}[breakable, size=fbox, boxrule=1pt, pad at break*=1mm,colback=cellbackground, colframe=cellborder]
\prompt{In}{incolor}{7}{\boxspacing}
\begin{Verbatim}[commandchars=\\\{\}]
\PY{k+kn}{import}\PY{+w}{ }\PY{n+nn}{torch}\PY{n+nn}{.}\PY{n+nn}{optim}\PY{+w}{ }\PY{k}{as}\PY{+w}{ }\PY{n+nn}{optim}

\PY{c+c1}{\PYZsh{} specify loss function (categorical cross\PYZhy{}entropy)}
\PY{n}{criterion} \PY{o}{=} \PY{n}{nn}\PY{o}{.}\PY{n}{CrossEntropyLoss}\PY{p}{(}\PY{p}{)}

\PY{c+c1}{\PYZsh{} specify optimizer}
\PY{n}{optimizer} \PY{o}{=} \PY{n}{optim}\PY{o}{.}\PY{n}{SGD}\PY{p}{(}\PY{n}{model}\PY{o}{.}\PY{n}{parameters}\PY{p}{(}\PY{p}{)}\PY{p}{,} \PY{n}{lr}\PY{o}{=}\PY{l+m+mf}{0.01}\PY{p}{)}

\PY{n}{optimizer} \PY{o}{=} \PY{n}{optim}\PY{o}{.}\PY{n}{Adam}\PY{p}{(}\PY{n}{model}\PY{o}{.}\PY{n}{parameters}\PY{p}{(}\PY{p}{)}\PY{p}{,} \PY{n}{lr}\PY{o}{=}\PY{l+m+mf}{0.001}\PY{p}{)} 
\end{Verbatim}
\end{tcolorbox}

    \subsection{\#\# Train the Network}\label{train-the-network}

Remember to look at how the training and validation loss decreases over
time; if the validation loss ever increases it indicates possible
overfitting.

The following is working code, but you are encouraged to make your own
adjustments and enhance the implementation.

    \begin{tcolorbox}[breakable, size=fbox, boxrule=1pt, pad at break*=1mm,colback=cellbackground, colframe=cellborder]
\prompt{In}{incolor}{8}{\boxspacing}
\begin{Verbatim}[commandchars=\\\{\}]
\PY{c+c1}{\PYZsh{} number of epochs to train the model, you decide the number}
\PY{n}{n\PYZus{}epochs} \PY{o}{=} \PY{l+m+mi}{5}

\PY{n}{valid\PYZus{}loss\PYZus{}min} \PY{o}{=} \PY{n}{np}\PY{o}{.}\PY{n}{inf} \PY{c+c1}{\PYZsh{} track change in validation loss}

\PY{k}{for} \PY{n}{epoch} \PY{o+ow}{in} \PY{n+nb}{range}\PY{p}{(}\PY{l+m+mi}{1}\PY{p}{,} \PY{n}{n\PYZus{}epochs}\PY{o}{+}\PY{l+m+mi}{1}\PY{p}{)}\PY{p}{:}

    \PY{c+c1}{\PYZsh{} keep track of training and validation loss}
    \PY{n}{train\PYZus{}loss} \PY{o}{=} \PY{l+m+mf}{0.0}
    \PY{n}{valid\PYZus{}loss} \PY{o}{=} \PY{l+m+mf}{0.0}
    
    \PY{c+c1}{\PYZsh{}\PYZsh{}\PYZsh{}\PYZsh{}\PYZsh{}\PYZsh{}\PYZsh{}\PYZsh{}\PYZsh{}\PYZsh{}\PYZsh{}\PYZsh{}\PYZsh{}\PYZsh{}\PYZsh{}\PYZsh{}\PYZsh{}\PYZsh{}\PYZsh{}}
    \PY{c+c1}{\PYZsh{} train the model \PYZsh{}}
    \PY{c+c1}{\PYZsh{}\PYZsh{}\PYZsh{}\PYZsh{}\PYZsh{}\PYZsh{}\PYZsh{}\PYZsh{}\PYZsh{}\PYZsh{}\PYZsh{}\PYZsh{}\PYZsh{}\PYZsh{}\PYZsh{}\PYZsh{}\PYZsh{}\PYZsh{}\PYZsh{}}
    \PY{n}{model}\PY{o}{.}\PY{n}{train}\PY{p}{(}\PY{p}{)}
    \PY{k}{for} \PY{n}{batch\PYZus{}idx}\PY{p}{,} \PY{p}{(}\PY{n}{data}\PY{p}{,} \PY{n}{target}\PY{p}{)} \PY{o+ow}{in} \PY{n+nb}{enumerate}\PY{p}{(}\PY{n}{train\PYZus{}loader}\PY{p}{)}\PY{p}{:}
        \PY{c+c1}{\PYZsh{} move tensors to GPU if CUDA is available}
        \PY{k}{if} \PY{n}{train\PYZus{}on\PYZus{}gpu}\PY{p}{:}
            \PY{n}{data}\PY{p}{,} \PY{n}{target} \PY{o}{=} \PY{n}{data}\PY{o}{.}\PY{n}{cuda}\PY{p}{(}\PY{p}{)}\PY{p}{,} \PY{n}{target}\PY{o}{.}\PY{n}{cuda}\PY{p}{(}\PY{p}{)}
        \PY{c+c1}{\PYZsh{} clear the gradients of all optimized variables}
        \PY{n}{optimizer}\PY{o}{.}\PY{n}{zero\PYZus{}grad}\PY{p}{(}\PY{p}{)}
        \PY{c+c1}{\PYZsh{} forward pass: compute predicted outputs by passing inputs to the model}
        \PY{n}{output} \PY{o}{=} \PY{n}{model}\PY{p}{(}\PY{n}{data}\PY{p}{)}
        \PY{c+c1}{\PYZsh{} calculate the batch loss}
        \PY{n}{loss} \PY{o}{=} \PY{n}{criterion}\PY{p}{(}\PY{n}{output}\PY{p}{,} \PY{n}{target}\PY{p}{)}
        \PY{c+c1}{\PYZsh{} backward pass: compute gradient of the loss with respect to model parameters}
        \PY{n}{loss}\PY{o}{.}\PY{n}{backward}\PY{p}{(}\PY{p}{)}
        \PY{c+c1}{\PYZsh{} perform a single optimization step (parameter update)}
        \PY{n}{optimizer}\PY{o}{.}\PY{n}{step}\PY{p}{(}\PY{p}{)}
        \PY{c+c1}{\PYZsh{} update training loss}
        \PY{n}{train\PYZus{}loss} \PY{o}{+}\PY{o}{=} \PY{n}{loss}\PY{o}{.}\PY{n}{item}\PY{p}{(}\PY{p}{)}\PY{o}{*}\PY{n}{data}\PY{o}{.}\PY{n}{size}\PY{p}{(}\PY{l+m+mi}{0}\PY{p}{)}
        
    \PY{c+c1}{\PYZsh{}\PYZsh{}\PYZsh{}\PYZsh{}\PYZsh{}\PYZsh{}\PYZsh{}\PYZsh{}\PYZsh{}\PYZsh{}\PYZsh{}\PYZsh{}\PYZsh{}\PYZsh{}\PYZsh{}\PYZsh{}\PYZsh{}\PYZsh{}\PYZsh{}\PYZsh{}\PYZsh{}\PYZsh{}    }
    \PY{c+c1}{\PYZsh{} validate the model \PYZsh{}}
    \PY{c+c1}{\PYZsh{}\PYZsh{}\PYZsh{}\PYZsh{}\PYZsh{}\PYZsh{}\PYZsh{}\PYZsh{}\PYZsh{}\PYZsh{}\PYZsh{}\PYZsh{}\PYZsh{}\PYZsh{}\PYZsh{}\PYZsh{}\PYZsh{}\PYZsh{}\PYZsh{}\PYZsh{}\PYZsh{}\PYZsh{}}
    \PY{n}{model}\PY{o}{.}\PY{n}{eval}\PY{p}{(}\PY{p}{)}
    \PY{k}{for} \PY{n}{batch\PYZus{}idx}\PY{p}{,} \PY{p}{(}\PY{n}{data}\PY{p}{,} \PY{n}{target}\PY{p}{)} \PY{o+ow}{in} \PY{n+nb}{enumerate}\PY{p}{(}\PY{n}{valid\PYZus{}loader}\PY{p}{)}\PY{p}{:}
        \PY{c+c1}{\PYZsh{} move tensors to GPU if CUDA is available}
        \PY{k}{if} \PY{n}{train\PYZus{}on\PYZus{}gpu}\PY{p}{:}
            \PY{n}{data}\PY{p}{,} \PY{n}{target} \PY{o}{=} \PY{n}{data}\PY{o}{.}\PY{n}{cuda}\PY{p}{(}\PY{p}{)}\PY{p}{,} \PY{n}{target}\PY{o}{.}\PY{n}{cuda}\PY{p}{(}\PY{p}{)}
        \PY{c+c1}{\PYZsh{} forward pass: compute predicted outputs by passing inputs to the model}
        \PY{n}{output} \PY{o}{=} \PY{n}{model}\PY{p}{(}\PY{n}{data}\PY{p}{)}
        \PY{c+c1}{\PYZsh{} calculate the batch loss}
        \PY{n}{loss} \PY{o}{=} \PY{n}{criterion}\PY{p}{(}\PY{n}{output}\PY{p}{,} \PY{n}{target}\PY{p}{)}
        \PY{c+c1}{\PYZsh{} update average validation loss }
        \PY{n}{valid\PYZus{}loss} \PY{o}{+}\PY{o}{=} \PY{n}{loss}\PY{o}{.}\PY{n}{item}\PY{p}{(}\PY{p}{)}\PY{o}{*}\PY{n}{data}\PY{o}{.}\PY{n}{size}\PY{p}{(}\PY{l+m+mi}{0}\PY{p}{)}
    
    \PY{c+c1}{\PYZsh{} calculate average losses}
    \PY{n}{train\PYZus{}loss} \PY{o}{=} \PY{n}{train\PYZus{}loss}\PY{o}{/}\PY{n+nb}{len}\PY{p}{(}\PY{n}{train\PYZus{}loader}\PY{o}{.}\PY{n}{sampler}\PY{p}{)}
    \PY{n}{valid\PYZus{}loss} \PY{o}{=} \PY{n}{valid\PYZus{}loss}\PY{o}{/}\PY{n+nb}{len}\PY{p}{(}\PY{n}{valid\PYZus{}loader}\PY{o}{.}\PY{n}{sampler}\PY{p}{)}
        
    \PY{c+c1}{\PYZsh{} print training/validation statistics }
    \PY{n+nb}{print}\PY{p}{(}\PY{l+s+s1}{\PYZsq{}}\PY{l+s+s1}{Epoch: }\PY{l+s+si}{\PYZob{}\PYZcb{}}\PY{l+s+s1}{ }\PY{l+s+se}{\PYZbs{}t}\PY{l+s+s1}{Training Loss: }\PY{l+s+si}{\PYZob{}:.6f\PYZcb{}}\PY{l+s+s1}{ }\PY{l+s+se}{\PYZbs{}t}\PY{l+s+s1}{Validation Loss: }\PY{l+s+si}{\PYZob{}:.6f\PYZcb{}}\PY{l+s+s1}{\PYZsq{}}\PY{o}{.}\PY{n}{format}\PY{p}{(}
        \PY{n}{epoch}\PY{p}{,} \PY{n}{train\PYZus{}loss}\PY{p}{,} \PY{n}{valid\PYZus{}loss}\PY{p}{)}\PY{p}{)}
    
    \PY{c+c1}{\PYZsh{} save model if validation loss has decreased}
    \PY{k}{if} \PY{n}{valid\PYZus{}loss} \PY{o}{\PYZlt{}}\PY{o}{=} \PY{n}{valid\PYZus{}loss\PYZus{}min}\PY{p}{:}
        \PY{n+nb}{print}\PY{p}{(}\PY{l+s+s1}{\PYZsq{}}\PY{l+s+s1}{Validation loss decreased (}\PY{l+s+si}{\PYZob{}:.6f\PYZcb{}}\PY{l+s+s1}{ \PYZhy{}\PYZhy{}\PYZgt{} }\PY{l+s+si}{\PYZob{}:.6f\PYZcb{}}\PY{l+s+s1}{).  Saving model ...}\PY{l+s+s1}{\PYZsq{}}\PY{o}{.}\PY{n}{format}\PY{p}{(}
        \PY{n}{valid\PYZus{}loss\PYZus{}min}\PY{p}{,}
        \PY{n}{valid\PYZus{}loss}\PY{p}{)}\PY{p}{)}
        \PY{n}{torch}\PY{o}{.}\PY{n}{save}\PY{p}{(}\PY{n}{model}\PY{o}{.}\PY{n}{state\PYZus{}dict}\PY{p}{(}\PY{p}{)}\PY{p}{,} \PY{l+s+s1}{\PYZsq{}}\PY{l+s+s1}{model\PYZus{}trained.pt}\PY{l+s+s1}{\PYZsq{}}\PY{p}{)}
        \PY{n}{valid\PYZus{}loss\PYZus{}min} \PY{o}{=} \PY{n}{valid\PYZus{}loss}
\end{Verbatim}
\end{tcolorbox}

    \begin{Verbatim}[commandchars=\\\{\}]
Epoch: 1        Training Loss: 1.330935         Validation Loss: 1.096905
Validation loss decreased (inf --> 1.096905).  Saving model {\ldots}
Epoch: 2        Training Loss: 0.961963         Validation Loss: 0.969708
Validation loss decreased (1.096905 --> 0.969708).  Saving model {\ldots}
Epoch: 3        Training Loss: 0.778138         Validation Loss: 0.898438
Validation loss decreased (0.969708 --> 0.898438).  Saving model {\ldots}
Epoch: 4        Training Loss: 0.614526         Validation Loss: 0.895830
Validation loss decreased (0.898438 --> 0.895830).  Saving model {\ldots}
Epoch: 5        Training Loss: 0.461253         Validation Loss: 0.962880
    \end{Verbatim}

    \subsubsection{Load the Model with the Lowest Validation
Loss}\label{load-the-model-with-the-lowest-validation-loss}

This is the model we will use for testing, which is the model we saved
in the last step

    \begin{tcolorbox}[breakable, size=fbox, boxrule=1pt, pad at break*=1mm,colback=cellbackground, colframe=cellborder]
\prompt{In}{incolor}{9}{\boxspacing}
\begin{Verbatim}[commandchars=\\\{\}]
\PY{n}{model}\PY{o}{.}\PY{n}{load\PYZus{}state\PYZus{}dict}\PY{p}{(}\PY{n}{torch}\PY{o}{.}\PY{n}{load}\PY{p}{(}\PY{l+s+s1}{\PYZsq{}}\PY{l+s+s1}{model\PYZus{}trained.pt}\PY{l+s+s1}{\PYZsq{}}\PY{p}{)}\PY{p}{)}
\end{Verbatim}
\end{tcolorbox}

            \begin{tcolorbox}[breakable, size=fbox, boxrule=.5pt, pad at break*=1mm, opacityfill=0]
\prompt{Out}{outcolor}{9}{\boxspacing}
\begin{Verbatim}[commandchars=\\\{\}]
<All keys matched successfully>
\end{Verbatim}
\end{tcolorbox}
        
    \subsection{\#\# Test the Trained
Network}\label{test-the-trained-network}

Test your trained model on previously unseen data! Remember we have
downloaded \texttt{train\_data} and \texttt{test\_data}. We will use
\texttt{test\_data} through \texttt{test\_loader}.

A ``good'' result will be a CNN that gets around 70\% (or more, try your
best!) accuracy on these test images.

The following is working code, but you are encouraged to make your own
adjustments and enhance the implementation.

    \begin{tcolorbox}[breakable, size=fbox, boxrule=1pt, pad at break*=1mm,colback=cellbackground, colframe=cellborder]
\prompt{In}{incolor}{10}{\boxspacing}
\begin{Verbatim}[commandchars=\\\{\}]
\PY{c+c1}{\PYZsh{} track test loss}
\PY{n}{test\PYZus{}loss} \PY{o}{=} \PY{l+m+mf}{0.0}
\PY{n}{class\PYZus{}correct} \PY{o}{=} \PY{n+nb}{list}\PY{p}{(}\PY{l+m+mf}{0.} \PY{k}{for} \PY{n}{i} \PY{o+ow}{in} \PY{n+nb}{range}\PY{p}{(}\PY{l+m+mi}{10}\PY{p}{)}\PY{p}{)}
\PY{n}{class\PYZus{}total} \PY{o}{=} \PY{n+nb}{list}\PY{p}{(}\PY{l+m+mf}{0.} \PY{k}{for} \PY{n}{i} \PY{o+ow}{in} \PY{n+nb}{range}\PY{p}{(}\PY{l+m+mi}{10}\PY{p}{)}\PY{p}{)}

\PY{n}{model}\PY{o}{.}\PY{n}{eval}\PY{p}{(}\PY{p}{)}
\PY{c+c1}{\PYZsh{} iterate over test data}
\PY{k}{for} \PY{n}{batch\PYZus{}idx}\PY{p}{,} \PY{p}{(}\PY{n}{data}\PY{p}{,} \PY{n}{target}\PY{p}{)} \PY{o+ow}{in} \PY{n+nb}{enumerate}\PY{p}{(}\PY{n}{test\PYZus{}loader}\PY{p}{)}\PY{p}{:}
    \PY{c+c1}{\PYZsh{} move tensors to GPU if CUDA is available}
    \PY{k}{if} \PY{n}{train\PYZus{}on\PYZus{}gpu}\PY{p}{:}
        \PY{n}{data}\PY{p}{,} \PY{n}{target} \PY{o}{=} \PY{n}{data}\PY{o}{.}\PY{n}{cuda}\PY{p}{(}\PY{p}{)}\PY{p}{,} \PY{n}{target}\PY{o}{.}\PY{n}{cuda}\PY{p}{(}\PY{p}{)}
    \PY{c+c1}{\PYZsh{} forward pass: compute predicted outputs by passing inputs to the model}
    \PY{n}{output} \PY{o}{=} \PY{n}{model}\PY{p}{(}\PY{n}{data}\PY{p}{)}
    \PY{c+c1}{\PYZsh{} calculate the batch loss}
    \PY{n}{loss} \PY{o}{=} \PY{n}{criterion}\PY{p}{(}\PY{n}{output}\PY{p}{,} \PY{n}{target}\PY{p}{)}
    \PY{c+c1}{\PYZsh{} update test loss }
    \PY{n}{test\PYZus{}loss} \PY{o}{+}\PY{o}{=} \PY{n}{loss}\PY{o}{.}\PY{n}{item}\PY{p}{(}\PY{p}{)}\PY{o}{*}\PY{n}{data}\PY{o}{.}\PY{n}{size}\PY{p}{(}\PY{l+m+mi}{0}\PY{p}{)}
    \PY{c+c1}{\PYZsh{} convert output probabilities to predicted class}
    \PY{n}{\PYZus{}}\PY{p}{,} \PY{n}{pred} \PY{o}{=} \PY{n}{torch}\PY{o}{.}\PY{n}{max}\PY{p}{(}\PY{n}{output}\PY{p}{,} \PY{l+m+mi}{1}\PY{p}{)}    
    \PY{c+c1}{\PYZsh{} compare predictions to true label}
    \PY{n}{correct\PYZus{}tensor} \PY{o}{=} \PY{n}{pred}\PY{o}{.}\PY{n}{eq}\PY{p}{(}\PY{n}{target}\PY{o}{.}\PY{n}{data}\PY{o}{.}\PY{n}{view\PYZus{}as}\PY{p}{(}\PY{n}{pred}\PY{p}{)}\PY{p}{)}
    \PY{n}{correct} \PY{o}{=} \PY{n}{np}\PY{o}{.}\PY{n}{squeeze}\PY{p}{(}\PY{n}{correct\PYZus{}tensor}\PY{o}{.}\PY{n}{numpy}\PY{p}{(}\PY{p}{)}\PY{p}{)} \PY{k}{if} \PY{o+ow}{not} \PY{n}{train\PYZus{}on\PYZus{}gpu} \PY{k}{else} \PY{n}{np}\PY{o}{.}\PY{n}{squeeze}\PY{p}{(}\PY{n}{correct\PYZus{}tensor}\PY{o}{.}\PY{n}{cpu}\PY{p}{(}\PY{p}{)}\PY{o}{.}\PY{n}{numpy}\PY{p}{(}\PY{p}{)}\PY{p}{)}
    \PY{c+c1}{\PYZsh{} calculate test accuracy for each object class}
    \PY{k}{for} \PY{n}{i} \PY{o+ow}{in} \PY{n+nb}{range}\PY{p}{(}\PY{n}{batch\PYZus{}size}\PY{p}{)}\PY{p}{:}
        \PY{n}{label} \PY{o}{=} \PY{n}{target}\PY{o}{.}\PY{n}{data}\PY{p}{[}\PY{n}{i}\PY{p}{]}
        \PY{n}{class\PYZus{}correct}\PY{p}{[}\PY{n}{label}\PY{p}{]} \PY{o}{+}\PY{o}{=} \PY{n}{correct}\PY{p}{[}\PY{n}{i}\PY{p}{]}\PY{o}{.}\PY{n}{item}\PY{p}{(}\PY{p}{)}
        \PY{n}{class\PYZus{}total}\PY{p}{[}\PY{n}{label}\PY{p}{]} \PY{o}{+}\PY{o}{=} \PY{l+m+mi}{1}

\PY{c+c1}{\PYZsh{} average test loss}
\PY{n}{test\PYZus{}loss} \PY{o}{=} \PY{n}{test\PYZus{}loss}\PY{o}{/}\PY{n+nb}{len}\PY{p}{(}\PY{n}{test\PYZus{}loader}\PY{o}{.}\PY{n}{dataset}\PY{p}{)}
\PY{n+nb}{print}\PY{p}{(}\PY{l+s+s1}{\PYZsq{}}\PY{l+s+s1}{Test Loss: }\PY{l+s+si}{\PYZob{}:.6f\PYZcb{}}\PY{l+s+se}{\PYZbs{}n}\PY{l+s+s1}{\PYZsq{}}\PY{o}{.}\PY{n}{format}\PY{p}{(}\PY{n}{test\PYZus{}loss}\PY{p}{)}\PY{p}{)}

\PY{k}{for} \PY{n}{i} \PY{o+ow}{in} \PY{n+nb}{range}\PY{p}{(}\PY{l+m+mi}{10}\PY{p}{)}\PY{p}{:}
    \PY{k}{if} \PY{n}{class\PYZus{}total}\PY{p}{[}\PY{n}{i}\PY{p}{]} \PY{o}{\PYZgt{}} \PY{l+m+mi}{0}\PY{p}{:}
        \PY{n+nb}{print}\PY{p}{(}\PY{l+s+s1}{\PYZsq{}}\PY{l+s+s1}{Test Accuracy of }\PY{l+s+si}{\PYZpc{}5s}\PY{l+s+s1}{: }\PY{l+s+si}{\PYZpc{}2d}\PY{l+s+si}{\PYZpc{}\PYZpc{}}\PY{l+s+s1}{ (}\PY{l+s+si}{\PYZpc{}2d}\PY{l+s+s1}{/}\PY{l+s+si}{\PYZpc{}2d}\PY{l+s+s1}{)}\PY{l+s+s1}{\PYZsq{}} \PY{o}{\PYZpc{}} \PY{p}{(}
            \PY{n}{classes}\PY{p}{[}\PY{n}{i}\PY{p}{]}\PY{p}{,} \PY{l+m+mi}{100} \PY{o}{*} \PY{n}{class\PYZus{}correct}\PY{p}{[}\PY{n}{i}\PY{p}{]} \PY{o}{/} \PY{n}{class\PYZus{}total}\PY{p}{[}\PY{n}{i}\PY{p}{]}\PY{p}{,}
            \PY{n}{np}\PY{o}{.}\PY{n}{sum}\PY{p}{(}\PY{n}{class\PYZus{}correct}\PY{p}{[}\PY{n}{i}\PY{p}{]}\PY{p}{)}\PY{p}{,} \PY{n}{np}\PY{o}{.}\PY{n}{sum}\PY{p}{(}\PY{n}{class\PYZus{}total}\PY{p}{[}\PY{n}{i}\PY{p}{]}\PY{p}{)}\PY{p}{)}\PY{p}{)}
    \PY{k}{else}\PY{p}{:}
        \PY{n+nb}{print}\PY{p}{(}\PY{l+s+s1}{\PYZsq{}}\PY{l+s+s1}{Test Accuracy of }\PY{l+s+si}{\PYZpc{}5s}\PY{l+s+s1}{: N/A (no training examples)}\PY{l+s+s1}{\PYZsq{}} \PY{o}{\PYZpc{}} \PY{p}{(}\PY{n}{classes}\PY{p}{[}\PY{n}{i}\PY{p}{]}\PY{p}{)}\PY{p}{)}

\PY{n+nb}{print}\PY{p}{(}\PY{l+s+s1}{\PYZsq{}}\PY{l+s+se}{\PYZbs{}n}\PY{l+s+s1}{Test Accuracy (Overall): }\PY{l+s+si}{\PYZpc{}2d}\PY{l+s+si}{\PYZpc{}\PYZpc{}}\PY{l+s+s1}{ (}\PY{l+s+si}{\PYZpc{}2d}\PY{l+s+s1}{/}\PY{l+s+si}{\PYZpc{}2d}\PY{l+s+s1}{)}\PY{l+s+s1}{\PYZsq{}} \PY{o}{\PYZpc{}} \PY{p}{(}
    \PY{l+m+mf}{100.} \PY{o}{*} \PY{n}{np}\PY{o}{.}\PY{n}{sum}\PY{p}{(}\PY{n}{class\PYZus{}correct}\PY{p}{)} \PY{o}{/} \PY{n}{np}\PY{o}{.}\PY{n}{sum}\PY{p}{(}\PY{n}{class\PYZus{}total}\PY{p}{)}\PY{p}{,}
    \PY{n}{np}\PY{o}{.}\PY{n}{sum}\PY{p}{(}\PY{n}{class\PYZus{}correct}\PY{p}{)}\PY{p}{,} \PY{n}{np}\PY{o}{.}\PY{n}{sum}\PY{p}{(}\PY{n}{class\PYZus{}total}\PY{p}{)}\PY{p}{)}\PY{p}{)}
\end{Verbatim}
\end{tcolorbox}

    \begin{Verbatim}[commandchars=\\\{\}]
Test Loss: 0.911915

Test Accuracy of airplane: 78\% (780/1000)
Test Accuracy of automobile: 84\% (849/1000)
Test Accuracy of  bird: 57\% (573/1000)
Test Accuracy of   cat: 55\% (554/1000)
Test Accuracy of  deer: 57\% (570/1000)
Test Accuracy of   dog: 59\% (599/1000)
Test Accuracy of  frog: 82\% (829/1000)
Test Accuracy of horse: 72\% (726/1000)
Test Accuracy of  ship: 75\% (754/1000)
Test Accuracy of truck: 74\% (744/1000)

Test Accuracy (Overall): 69\% (6978/10000)
    \end{Verbatim}

    \subsubsection{Visualize Sample Test
Results}\label{visualize-sample-test-results}

The following is working code, but you are encouraged to make your own
adjustments and enhance the visualization.

    \begin{tcolorbox}[breakable, size=fbox, boxrule=1pt, pad at break*=1mm,colback=cellbackground, colframe=cellborder]
\prompt{In}{incolor}{11}{\boxspacing}
\begin{Verbatim}[commandchars=\\\{\}]
\PY{c+c1}{\PYZsh{} obtain one batch of test images}
\PY{n}{dataiter} \PY{o}{=} \PY{n+nb}{iter}\PY{p}{(}\PY{n}{test\PYZus{}loader}\PY{p}{)}
\PY{n}{images}\PY{p}{,} \PY{n}{labels} \PY{o}{=} \PY{n+nb}{next}\PY{p}{(}\PY{n}{dataiter}\PY{p}{)}
\PY{n}{images}\PY{o}{.}\PY{n}{numpy}\PY{p}{(}\PY{p}{)}

\PY{c+c1}{\PYZsh{} move model inputs to cuda, if GPU available}
\PY{k}{if} \PY{n}{train\PYZus{}on\PYZus{}gpu}\PY{p}{:}
    \PY{n}{images} \PY{o}{=} \PY{n}{images}\PY{o}{.}\PY{n}{cuda}\PY{p}{(}\PY{p}{)}

\PY{c+c1}{\PYZsh{} get sample outputs}
\PY{n}{output} \PY{o}{=} \PY{n}{model}\PY{p}{(}\PY{n}{images}\PY{p}{)}
\PY{c+c1}{\PYZsh{} convert output probabilities to predicted class}
\PY{n}{\PYZus{}}\PY{p}{,} \PY{n}{preds\PYZus{}tensor} \PY{o}{=} \PY{n}{torch}\PY{o}{.}\PY{n}{max}\PY{p}{(}\PY{n}{output}\PY{p}{,} \PY{l+m+mi}{1}\PY{p}{)}
\PY{n}{preds} \PY{o}{=} \PY{n}{np}\PY{o}{.}\PY{n}{squeeze}\PY{p}{(}\PY{n}{preds\PYZus{}tensor}\PY{o}{.}\PY{n}{numpy}\PY{p}{(}\PY{p}{)}\PY{p}{)} \PY{k}{if} \PY{o+ow}{not} \PY{n}{train\PYZus{}on\PYZus{}gpu} \PY{k}{else} \PY{n}{np}\PY{o}{.}\PY{n}{squeeze}\PY{p}{(}\PY{n}{preds\PYZus{}tensor}\PY{o}{.}\PY{n}{cpu}\PY{p}{(}\PY{p}{)}\PY{o}{.}\PY{n}{numpy}\PY{p}{(}\PY{p}{)}\PY{p}{)}

\PY{c+c1}{\PYZsh{} plot the images in the batch, along with predicted and true labels}
\PY{n}{fig} \PY{o}{=} \PY{n}{plt}\PY{o}{.}\PY{n}{figure}\PY{p}{(}\PY{n}{figsize}\PY{o}{=}\PY{p}{(}\PY{l+m+mi}{25}\PY{p}{,} \PY{l+m+mi}{4}\PY{p}{)}\PY{p}{)}
\PY{k}{for} \PY{n}{idx} \PY{o+ow}{in} \PY{n}{np}\PY{o}{.}\PY{n}{arange}\PY{p}{(}\PY{l+m+mi}{20}\PY{p}{)}\PY{p}{:}
    \PY{n}{ax} \PY{o}{=} \PY{n}{fig}\PY{o}{.}\PY{n}{add\PYZus{}subplot}\PY{p}{(}\PY{l+m+mi}{2}\PY{p}{,} \PY{n+nb}{int}\PY{p}{(}\PY{l+m+mi}{20}\PY{o}{/}\PY{l+m+mi}{2}\PY{p}{)}\PY{p}{,} \PY{n}{idx}\PY{o}{+}\PY{l+m+mi}{1}\PY{p}{,} \PY{n}{xticks}\PY{o}{=}\PY{p}{[}\PY{p}{]}\PY{p}{,} \PY{n}{yticks}\PY{o}{=}\PY{p}{[}\PY{p}{]}\PY{p}{)}
    \PY{n}{imshow}\PY{p}{(}\PY{n}{images}\PY{p}{[}\PY{n}{idx}\PY{p}{]}\PY{p}{)}
    \PY{n}{ax}\PY{o}{.}\PY{n}{set\PYZus{}title}\PY{p}{(}\PY{l+s+s2}{\PYZdq{}}\PY{l+s+si}{\PYZob{}\PYZcb{}}\PY{l+s+s2}{ (}\PY{l+s+si}{\PYZob{}\PYZcb{}}\PY{l+s+s2}{)}\PY{l+s+s2}{\PYZdq{}}\PY{o}{.}\PY{n}{format}\PY{p}{(}\PY{n}{classes}\PY{p}{[}\PY{n}{preds}\PY{p}{[}\PY{n}{idx}\PY{p}{]}\PY{p}{]}\PY{p}{,} \PY{n}{classes}\PY{p}{[}\PY{n}{labels}\PY{p}{[}\PY{n}{idx}\PY{p}{]}\PY{p}{]}\PY{p}{)}\PY{p}{,}
                 \PY{n}{color}\PY{o}{=}\PY{p}{(}\PY{l+s+s2}{\PYZdq{}}\PY{l+s+s2}{green}\PY{l+s+s2}{\PYZdq{}} \PY{k}{if} \PY{n}{preds}\PY{p}{[}\PY{n}{idx}\PY{p}{]}\PY{o}{==}\PY{n}{labels}\PY{p}{[}\PY{n}{idx}\PY{p}{]}\PY{o}{.}\PY{n}{item}\PY{p}{(}\PY{p}{)} \PY{k}{else} \PY{l+s+s2}{\PYZdq{}}\PY{l+s+s2}{red}\PY{l+s+s2}{\PYZdq{}}\PY{p}{)}\PY{p}{)}
\end{Verbatim}
\end{tcolorbox}

    \begin{center}
    \adjustimage{max size={0.9\linewidth}{0.9\paperheight}}{output_21_0.png}
    \end{center}
    { \hspace*{\fill} \\}
    

    % Add a bibliography block to the postdoc
    
    
    
\end{document}
